\documentclass{article}
\usepackage{amsmath,amssymb,braket,url,hyperref,booktabs,graphicx}
\usepackage[style=nature]{biblatex}
\addbibresource{13_ref.bib}
\newcommand{\bfit}[1]{\textit{\textbf{#1}}}
\begin{document}
\medskip

\begin{flushright}
\textbf{\LARGE Part III\\
Superconducting Qubit architecture and\\
Hardware}
\end{flushright}

\newpage
\noindent
\textbf{\Large Chapter 13\\
Lagrangian Mechanics and Hamiltonian\\
Mechanics}
\\[30pt]
\textbf{\large 13.1 Introduction}
\\\\
Most of us have learned the basics of Newtonian mechanics. Newtonian mechanics
is also called \textit{vectorial mechanics} because it studies the motion of bodies under the influence
of vector euantities such as $force$. However, Newtonian mechanics is not convenient in solving certain
problems. There are other frameworks in theoretical physics called analytical mechanics such as 
\textbf{Lagrangian mechanics} and \textbf{Hamiltonian mechanics}. They are equivalent to Newtonian mechanics and
they use scalar quantities such as the kinetic energy and potential energy of a system to derive
the equations of motion of the system. In many problems, they appear to be more elegant and succincy than 
Newtonian mechanics. More importantly, the concpets in analytical mechanics can be \textit{generalized}
to hyperspace/phase space, in which we do not live. Moreover, Hamiltonian mechanics allow us to transition from
classical mechanics to quantum mechanics more "smoothly." In this chapter, we will learn the 
\textit{skills} of using Lagrangian and Hamiltonian mechanics. Readers are expected to learn the rules only. Readers
may refer to \cite{goldstein2001pearson} if they are interested in having a deeper appreciation of analytical mechanics.
\\[20pt]
\bfit{\large 13.1.1 Learning Outcomes}
\\\\
Be able to write down the Lagrangian and Hamiltonian of a given physical system;
be able to derive the equation of motion of a system based on its Lagrangian and Hamiltonian.
\\[20pt]
\bfit{\large 13.1.2 Teaching Videos}
\\\\
$\bullet$ Search for Ch13 in this playlist

- \url{https://tinyurl.com/3yhze3jn}\\\\
$\bullet$ Other Videos

- \url{https://youtu.be/Ydj2hintCkc}

- \url{https://youtu.be/u2SgXmf2SvQ}

-\url{https://youtu.be/IdSF_064ZSo}
\\[30pt]
\textbf{\large 13.2 Lagrangian Mechanics}
\\\\
\bfit{\large 13.2.1 Generalized Coordinates and Belocities}
\\\\
For the purposes in the following chapters, we only consider point particles, conservative forces,
and non-relativistic mechanics. Let us consider a system comprised of $N$ 
particle. We know that if the coordinates of each particle and the velocity of each particle are known at a 
given  time, the system has a well-defined state. This is because the
acceleration of a particle depends on the force exerted on it. 
And the force is the spatial derivative of its potential, which is a function of its
coordinates. Therefore, if we know their positions and velocities, we know
their accelerations and, thus, can deduce their past and future states.

For $N$ particles, inour real space, there are 3$N$ independent coordinates
due to the three orthogonal directions. Therefore, they are the collection of 
$\vec{q}=\{ q_1,q_2,\cdots,q_{3N} \}$, where we write it as a 3$N$-dimensional vector.
Similarly, it has 3$N$ independent velocities, $\vec{\dot{q}}=\{ \dot{q_1},\dot{q_2},\cdots,\dot{q_{3N}} \}$, where
\begin{equation}\label{eq 13.1}
    \dot{q_i}=\frac{dq_i}{dt}. \tag{13.1}
\end{equation}

Besides using real spatial coordinates and velocities to uniquely determine the
state of a system, one may also use other 3$N$ quantities to determine its coordinates as long as
they also give the system 3$N$ degrees of freedom \cite{Landau}. Such quantities are called
the \textbf{generalized coordinates}. The time derivatives (Eq.(\ref{eq 13.1})) of the
generalized coordinates are called the \textbf{generalized velocities}. For the formalism we will
discuss later, it is easier to think and understand using spatial coordinates and velocities but
we need to keep in mind and accept the fact that they are applicable to generalized coordinates and velocities, too.
\\[20pt]
\bfit{\large 13.2.2 Lagrangian and Lagrange's Equations}
\\\\
We will first introduce a neqy quantity called \textbf{Lagrangian}, $\mathcal{L}$, which is defined as,
\begin{equation}\label{eq 13.2}
    \mathcal{L}=T-V, \tag{13.2}
\end{equation}  
where T and V and the \textbf{kinetic energy} and \textbf{potential energy} of the system,
respectively. Naturally, $\mathcal{L}$ has the unit of energy. Since T is a function of velocities,
$\vec{\dot{q}}$, and V is a function of coordinates, $\vec{q}$, $\mathcal{L}$ is a function of both 
velocities and coordinates. Of course, they are all functions of time, $t$. Therefore,
\begin{align*}\label{eq 13.3}
    \mathcal{L}&=T(\vec{\dot{q}},t)-V(\vec{\dot{q}},t),\\
    &=\mathcal{L}(\vec{q},\vec{\dot{q}},t),\\
    &=\mathcal{L}(q_1,q_2,\cdots,\dot{q_1},\dot{q_2},\cdots,t). \tag{13.3}
\end{align*}

It is given that one can derive the \textbf{equations of motion} of the system by solving 
the \textbf{Lagrangian's equation} \cite{goldstein2001pearson},
\begin{equation}\label{eq 13.4}
    \frac{d}{dt}\left(\frac{\partial \mathcal{L}}{\partial \dot{q_i}}\right)
    -\frac{\partial \mathcal{L}}{\partial q_i}=0,\tag{13.4}
\end{equation}
fot $i$ from 1 to 3$N$. Lagrange's equation contain the Lagrangian of the system with
the \textit{coordinates and velocities being the independent variables}. This forms the basics of
Lagrangian mechanics. It should be noted that when working with Lagrangian 
mechanics, one needs to make sure to \textbf{express the Lagrangian explicitly as a
function of coordinates and velocities}. Of course, this includes the \textit{generalized} coordinates
and velocities.

It should also be noted that the Lagrangian of a system is \textit{not unique}. As long as it 
gives the correct equations of motion through Eq. (\ref{eq 13.4}), it is a valid Lagrangian.

In this book, we take Lagrangian's equation as given like how we trust $F=ma$ when we study
\textbf{Newtonian mechanics}. But Lagrangian's equation can be derived from a more
fundamental principle, namely, the \textbf{principle of least action} or \textbf{Hamiltonian's principle}
\cite{Landau}. The principle defines \textbf{action} $S$, as,
]\begin{equation}\label{eq 13.5}
    S=\int_{t_1}^{t_2}\mathcal{L}(\vec{q},\vec{\dot{q}},t)  \,dt.\tag{13.5} 
\end{equation}

The action, $S$, integrates the Lagrangian of a system from time $t_1$ to time $t_2$ and
mandates that the system should evolve from time $t_1$ to time $t_2$ in a way such that $S$
is minimal based on which the Lagrange's equation to Eq. (\ref{eq 13.4}) are derived \cite{Landau}.

Again, for the purpose of this book, we just need to learn the skills to construct the Lagrangian and solve
the Lagrange's equation of a given system. Let us look at two examples.
\\\\
\textbf{Example 13.1} Find the equations of motion of a free particle with mass, $m$.
Assume that the particle is moving in the $\hat{x}$-direction with speed $v$ at time $t_0$.

We already know from Newton's first law that a free particle will keep moving at a constant
speed. Let us see if we will obtain the same result by using Lagrangian mechanics.

Set $\vec{q}= \{ q_1=x, q_2=y, q_3=z \}$. Its velocity is $\vec{\dot{q}}=\{\dot{q_1},\dot{q_2},\dot{q_3} \}$.
As a free particle, there is no external force and thus it experiences a constant potential energy that can be set to a constant $\phi$. 
Therefore, its kinetic energy is
\begin{equation}\label{eq 13.6}
    T=\frac{1}{2}m\dot{q_1}^2+\frac{1}{2}m\dot{q_2}^2+\frac{1}{2}m\dot{q_3}^2,\tag{13.6}
\end{equation}
and its potential energy is
\begin{equation}\label{eq 13.7}
    V=\phi.\tag{13.7}
\end{equation}
The Lagrangian of the system is
\begin{align*}\label{eq 13.8}
    \mathcal{L}&=T-V,\\
    &=\frac{1}{2}m\dot{q_1}^2+\frac{1}{2}m\dot{q_2}^2+\frac{1}{2}m\dot{q_3}^2-\phi.\tag{13.8}
\end{align*}
To find its equation of motion, we solve Lagrange's equations in Eq. (\ref{eq 13.4}). 
Note that there are three coordinates (for $i=1$ to $i=3$ with $q_1=x,q_2=y,q_3=z$).
Therefore, we have three equations,
\begin{align*}\label{eq 13.9}
     \frac{d}{dt}\left(\frac{\partial \mathcal{L}}{\partial \dot{q_1}}\right)
    -\frac{\partial \mathcal{L}}{\partial q_1}=0,\tag{13.9}\\[5pt]
     \frac{d}{dt}\left(\frac{\partial \mathcal{L}}{\partial \dot{q_2}}\right)
    -\frac{\partial \mathcal{L}}{\partial q_2}=0,\tag{13.10}\\[5pt]
     \frac{d}{dt}\left(\frac{\partial \mathcal{L}}{\partial \dot{q_3}}\right)
    -\frac{\partial \mathcal{L}}{\partial q_3}=0,\tag{13.11}\\
\end{align*}
The equation involve partial derivatives with respect to $q_1,q_2,q_3,\dot{q_1},\dot{q_2}$, and $\dot{q_3}$.
But none of the erms in Eq. (\ref{eq 13.8}) depends on $q_1,q_2,$ and $q_3$ as $\phi$ is a constant.
$\mathcal{L}$ only depends on $\dot{q_1},\dot{q_2},$ and $\dot{q_3}$. Therefore, the three Lagrange's equation becomes,
\begin{align*}\label{eq 13.12}
      \frac{d}{dt}\left(\frac{\partial \mathcal{L}}{\partial \dot{q_1}}\right)
    -0=0,\tag{13.12}\\[5pt]
     \frac{d}{dt}\left(\frac{\partial \mathcal{L}}{\partial \dot{q_2}}\right)
    -0=0,\tag{13.13}\\[5pt]
     \frac{d}{dt}\left(\frac{\partial \mathcal{L}}{\partial \dot{q_3}}\right)
    -0=0,\tag{13.14}\\
\end{align*}
Let us only solve Eq. (\ref{eq 13.12}),
\begin{align*}\label{eq 13.15}
    \frac{d}{dt}\left(\frac{\partial \mathcal{L}}{\partial \dot{q_1}}\right)&=0,\\
    \frac{d}{dt}\left(\frac{\partial(\frac{1}{2}m\dot{q_1}^2+\frac{1}{2}m\dot{q_2}^2+
    \frac{1}{2}m\dot{q_3}^2-\phi)}{\partial \dot{q_1}}\right)&=0,\\
    \frac{d(m\dot{q_1})}{dt}&=0,\\
    \ddot{q_1}&=0.\tag{13.15}
\end{align*}
$\ddot{q_1}$ is the time derivative of velocity, which is the acceleration in the $\hat{x}$ direction.
It means the particle will move at a constant velocity in the $\hat{x}$ direction. Since at 
$t=t_0,\dot{q_1}=v$, then it will be moving at speed $v$ in the $\hat{x}$ direction forever. Similarly,
$\ddot{q_2}=\ddot{q_3}=0$; therefore, $q_2=q_3=0$ at all time.

Using Lagrangian mechanics, we obtain the same conclusion as Newton's first law.
\\\\
\textbf{Example 13.2} Find the euqation of motion of a mass, $m$, attached to a fixed wall through a spring with a spring 
with a spring constant of $k$ (Fig. 13.1). This is the famous \textbf{simple harmonic oscillator (SHO)}
problem.

This is a 1D system
\printbibliography
\end{document}