\documentclass{article}
\usepackage{amsmath,amssymb,braket,url,hyperref,booktabs,graphicx}
\usepackage[style=nature]{biblatex}
\addbibresource{13_ref.bib}
\newcommand{\bfit}[1]{\textit{\textbf{#1}}}
\begin{document}
\medskip

\begin{flushright}
\textbf{\LARGE Part III\\
Superconducting Qubit architecture and\\
Hardware}
\end{flushright}

\newpage
\noindent
\textbf{\Large Chapter 13\\
Lagrangian Mechanics and Hamiltonian\\
Mechanics}
\\[30pt]
\textbf{\large 13.1 Introduction}
\\\\
Most of us have learned the basics of Newtonian mechanics. Newtonian mechanics
is also called \textit{vectorial mechanics} because it studies the motion of bodies under the influence
of vector euantities such as $force$. However, Newtonian mechanics is not convenient in solving certain
problems. There are other frameworks in theoretical physics called analytical mechanics such as 
\textbf{Lagrangian mechanics} and \textbf{Hamiltonian mechanics}. They are equivalent to Newtonian mechanics and
they use scalar quantities such as the kinetic energy and potential energy of a system to derive
the equations of motion of the system. In many problems, they appear to be more elegant and succincy than 
Newtonian mechanics. More importantly, the concpets in analytical mechanics can be \textit{generalized}
to hyperspace/phase space, in which we do not live. Moreover, Hamiltonian mechanics allow us to transition from
classical mechanics to quantum mechanics more "smoothly." In this chapter, we will learn the 
\textit{skills} of using Lagrangian and Hamiltonian mechanics. Readers are expected to learn the rules only. Readers
may refer to \cite{goldstein2001pearson} if they are interested in having a deeper appreciation of analytical mechanics.
\\[20pt]
\bfit{\large 13.1.1 Learning Outcomes}
\\\\
Be able to write down the Lagrangian and Hamiltonian of a given physical system;
be able to derive the equation of motion of a system based on its Lagrangian and Hamiltonian.
\\[20pt]
\bfit{\large 13.1.2 Teaching Videos}
\\\\
$\bullet$ Search for Ch13 in this playlist

- \url{https://tinyurl.com/3yhze3jn}\\\\
$\bullet$ Other Videos

- \url{https://youtu.be/Ydj2hintCkc}

- \url{https://youtu.be/u2SgXmf2SvQ}

-\url{https://youtu.be/IdSF_064ZSo}
\\[30pt]
\textbf{\large 13.2 Lagrangian Mechanics}
\\\\
\bfit{\large 13.2.1 Generalized Coordinates and Belocities}
\\\\
For the purposes in the following chapters, we only consider point particles, conservative forces,
and non-relativistic mechanics. Let us consider a system comprised of $N$ 
particle. We know that if the coordinates of each particle and the velocity of each particle are known at a 
given  time, the system has a well-defined state. This is because the
acceleration of a particle depends on the force exerted on it. 
And the force is the spatial derivative of its potential, which is a function of its
coordinates. Therefore, if we know their positions and velocities, we know
their accelerations and, thus, can deduce their past and future states.

For $N$ particles, inour real space, there are 3$N$ independent coordinates
due to the three orthogonal directions. Therefore, they are the collection of 
$\vec{q}=\{ q_1,q_2,\cdots,q_{3N} \}$, where we write it as a 3$N$-dimensional vector.
Similarly, it has 3$N$ independent velocities, $\vec{\dot{q}}=\{ \dot{q_1},\dot{q_2},\cdots,\dot{q_{3N}} \}$, where
\begin{equation}\label{eq 13.1}
    \dot{q_i}=\frac{dq_i}{dt}. \tag{13.1}
\end{equation}

Besides using real spatial coordinates and velocities to uniquely determine the
state of a system, one may also use other 3$N$ quantities to determine its coordinates as long as
they also give the system 3$N$ degrees of freedom \cite{Landau}. Such quantities are called
the \textbf{generalized coordinates}. The time derivatives (Eq.(\ref{eq 13.1})) of the
generalized coordinates are called the \textbf{generalized velocities}. For the formalism we will
discuss later, it is easier to think and understand using spatial coordinates and velocities but
we need to keep in mind and accept the fact that they are applicable to generalized coordinates and velocities, too.
\\[20pt]
\bfit{\large 13.2.2 Lagrangian and Lagrange's Equations}
\\\\
We will

\printbibliography
\end{document}