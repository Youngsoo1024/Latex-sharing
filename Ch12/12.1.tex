\documentclass{article}
\usepackage{amsmath,amssymb,braket,url,hyperref,booktabs,graphicx}
\usepackage[style=nature]{biblatex}
\addbibresource{12_ref.bib}
\newcommand{\bfit}[1]{\textit{\textbf{#1}}}
\begin{document}
\medskip
\textbf{\Large Chapter 12\\
Electron Spin Qubit in Semicoductor-1-Qubit and 2-Qubit Gates}\\\\\\
\medskip
\textbf{\large 12.1 Introduction}\\
In Chap.11, we showed how to implement a qubit using electron spin on a silicon
substrate. We also demonetrated how to initialize and measure a qubit. In this chapter,
we will sudy how to perform a universal 1-qubit gate and a 2-qubit entnaglement gate to fulfill
the last two DiVincenzo's criteria (Sect. 1.3). In Chap.10, we showed
that by applying a vertical DC magnetic field and a rotating  horizontal magnetic field and
then \textit{working in the rotating frame}, we would be able to rotate any state on Bloch sphere
about any vector. This allows us to build a universal 1-qubit gate (Section 27.4 in \cite{WongHuiYong})
However, in the literature, many silicon qubits are still implemented with the setup in Chap. 9 which means
that the qubit is places in a vertical DC magnetic field and a perturbating and linearly oscillating horizontal
magnetic field. This is what we will use in this chapter. We will first summarize an experimental paper on how
it implements 1-qubit gate. Then we will discuss the implementation of a 2-qubit entnaglement gate with an example.


\printbibliography
\end{document}
