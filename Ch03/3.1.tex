\documentclass{article}
\usepackage{amsmath}
\usepackage{hyperref}
\usepackage{braket}
\begin{document}

\textbf{\Large Chapter 3} \\ 
\textbf{\Large Linear Algebra-Operators, Matrices, and Quantum Gates}
\\
\\
\\
\textbf{\large 3.1 Introduction}
\\
\\
Vectors are the fundamentals in a Hilbert space. They represent the state of a physical system corresponding to that
Hilbert space. But a state or a vector is \textit{passive}. What is interesting is the application of an operator to rotate
a vector  in the space or, in other words, to transform the state. An operator corresponds to a matrix.
In this chapter, we will review the concepts of eigenvalue, eigenvector, Hermitian matrix, and unitary matrix.
We will also review how to construct a projection operator and a unitary transformation matrix.
Then we will revisit the meaning of a measurement in a quantum system using the new knowledge we have learned. Finally,
we will discuss how to perform a tensor product for matrices.
\\
\\
\textbf{\textit{3.1.1 Learning Outcomes}}
\\
\\
Understand the definitions of HErmitian matrix, unitary matrix, and project matrix;
able to transform vectors and matrices from one basis to another; appreciate
that applying an operator to a vector is to rotate the vector in its Hilbert space.
\\
\\
\textbf{\textit{3.1.2 Teaching Videos}}
\\
\\
$\cdot$ Search for Ch3 in this playlist\\
- \url{http://tinyurl.com/3yhze3jn}\\
\\
$\cdot$ Other Vidoes\\
- \url{http://youtu.be/Wrmigi645J4}\\
- \url{http://youtu.be/Z_fXDssH2JA}\\
\\
\\
\textbf{\large 3.2 Operators}
\\
\\
An \textbf{operator} in a Hilbert space maps the vectors in one space to another space.
However, sometimes, the second space is the same as the first one. In this case,
an operator can be considered to be rotating a vector in its space. Here we will consider
the operator that map the vectors to the same space. For example, an operator \textbf{\textit{M}} 
applied to vector $\ket{\alpha^\prime}$,

\begin{equation} \label{eq 3.1}
    \ket{\alpha^\prime}= \textbf{\textit{M}} \ket{\alpha}. \tag{3.1}
\end{equation}


If the vector is represented as a colum vector, it is natural that the operator must be a matrix.

The operators are linear and observe the distribution law,

\begin{equation} \label{eq 3.2}
    \textbf{\textit{M}} (\textit{C}_{\alpha}\ket{\alpha}+ \textit{C}_{\beta}\ket{\beta})=\textit{C}_{\alpha}
\textbf{\textit{M}} \ket{\alpha}+\textit{C}_{\beta}\textbf{\textit{M}}\ket{\beta}, \tag{3.2}
\end{equation}

where $\textit{C}_{\alpha}$ and $\textit{C}_{\beta}$ are complex scalra and $\ket{\beta}$ is another vector.
\\
\\
\textit{\textbf{\large 3.2.1 Dual Correspondence}}
\\
\\
We introduced the concept of \textbf{dual correspondence} in Chap. 2 that every vector $\ket{\alpha}$
in the \textit{ket} space has a correspondence vector $\bra{\alpha}$ in the \textit{bra} space. Therefore,
there is a corresponding \textit{bra} version for $\bra{\alpha^{\prime}}$, i.e., $\bra{\alpha^{\prime}}$,
in Eq. (\ref{eq 3.1}),

\begin{equation}
    \bra{\alpha^{\prime}} \Leftrightarrow \ket{\alpha^{\prime}} \tag{3.3}
\end{equation}

Is there a dual correspondence for the \textit{ket}-space operator \textit{\textbf{M}}?
Yes, it is the \textbf{adjoint} of \textit{\textbf{M}}, i.e.. $\textit{\textbf{M}}^{\dagger}$.
The adjoint of a matrix is just the conjugate transpose of that matrix. This is the same as how we find
the \textit{bra} version of a vector (e.g., Eq.(2.19)). Therefore, Eq. (3.3) can be written as,

\begin{equation} \label{eq 3.4}
    \bra{\alpha}\textit{\textbf{M}}^{\dagger}\Leftrightarrow\textit{\textbf{M}}\ket{\alpha} \tag{3.4}
\end{equation}
\\
Note that the operator is applied form the \textbf{right} of the bra vector.
This can be easily appreciated by recognizing that a bra vector is a row vector instead of a column vector.


The dual equation of Eq. (\ref{eq 3.2}) is thus,

\begin{equation} \label{eq 3.5}
     (\textit{C}_{\alpha}^*\bra{\alpha}+ \textit{C}_{\beta}^*\bra{\beta})\textbf{\textit{M}}^{\dagger}
     =\textit{C}_{\alpha}^* \bra{\alpha}\textbf{\textit{M}}^{\dagger}+\textit{C}_{\beta}^*\bra{\beta}\textbf{\textit{M}}^{\dagger}. \tag{3.5}    
\end{equation}

Note that complex conjugate is taken for scalars $\textit{C}_{\alpha}$ and $\textit{C}_{\beta}$ in the \textit{bra} version.


Based on the first line Eq. (2.6), we can further derive that

\begin{align} \label{eq 3.6}
    \begin{split}
        \braket{\beta|\alpha^\prime}=\braket{\alpha^\prime|\beta}^*\\
\braket{\beta|\textit{\textbf{M}}|\alpha}=\braket{\alpha|\textit{\textbf{M}}^{\dagger}|\beta}
    \end{split}\tag{3.6}
\end{align}


These are some useful equations we will use in quantum mechanics.
\end{document}